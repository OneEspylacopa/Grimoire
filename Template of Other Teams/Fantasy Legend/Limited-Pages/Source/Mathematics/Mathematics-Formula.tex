\section{常用数学公式}
	\subsection{求和公式}
		\begin{enumerate}\setlength{\itemsep}{-\itemsep}
			\item $\sum_{k=1}^{n}(2k-1)^2 = \frac{n(4n^2-1)}{3}	$
			\item $\sum_{k=1}^{n}k^3 = [\frac{n(n+1)}{2}]^2	$
			\item $\sum_{k=1}^{n}(2k-1)^3 = n^2(2n^2-1)	$
			\item $\sum_{k=1}^{n}k^4 = \frac{n(n+1)(2n+1)(3n^2+3n-1)}{30}  $
			\item $\sum_{k=1}^{n}k^5 = \frac{n^2(n+1)^2(2n^2+2n-1)}{12}	$
			\item $\sum_{k=1}^{n}k(k+1) = \frac{n(n+1)(n+2)}{3}	$
			\item $\sum_{k=1}^{n}k(k+1)(k+2) = \frac{n(n+1)(n+2)(n+3)}{4} $
			\item $\sum_{k=1}^{n}k(k+1)(k+2)(k+3) = \frac{n(n+1)(n+2)(n+3)(n+4)}{5} $
		\end{enumerate}
	\subsection{斐波那契数列}
		\begin{enumerate}\setlength{\itemsep}{-\itemsep}
			\item $fib_0=0, fib_1=1, fib_n=fib_{n-1}+fib_{n-2}$
			\item $fib_{n+2} \cdot fib_n-fib_{n+1}^2=(-1)^{n+1}$
			\item $fib_{-n}=(-1)^{n-1}fib_n$
			\item $fib_{n+k}=fib_k \cdot fib_{n+1}+fib_{k-1} \cdot fib_n$
			\item $gcd(fib_m, fib_n)=fib_{gcd(m, n)}$
			\item $fib_m|fib_n^2\Leftrightarrow nfib_n|m$
		\end{enumerate}
	\subsection{错排公式}
		\begin{enumerate}\setlength{\itemsep}{-\itemsep}
			\item $D_n = (n-1)(D_{n-2}-D_{n-1})$
			\item $D_n = n! \cdot (1-\frac{1}{1!}+\frac{1}{2!}-\frac{1}{3!}+\ldots+\frac{(-1)^n}{n!})$
		\end{enumerate}
	\subsection{莫比乌斯函数}
		$\mu(n) = \begin{cases}
			1 & \text{若}n=1\\
			(-1)^k & \text{若}n\text{无平方数因子,且}n = p_1p_2\dots p_k\\
			0 & \text{若}n\text{有大于}1\text{的平方数因数}
		\end{cases}$
		\par
		$\sum_{d|n}{\mu(d)} = \begin{cases}
			1 & \text{若}n=1\\
			0 & \text{其他情况}
		\end{cases}$
		\par
		$g(n) = \sum_{d|n}{f(d)} \Leftrightarrow f(n) = \sum_{d|n}{\mu(d)g(\frac{n}{d})}$\par
		$g(x) = \sum_{n=1}^{[x]}f(\frac{x}{n}) \Leftrightarrow f(x) = \sum_{n=1}^{[x]}{\mu(n)g(\frac{x}{n})}$
	\subsection{Burnside引理}
		设$G$是一个有限群,作用在集合$X$上。对每个$g$属于$G$,令$X^g$表示$X$中在$g$作用下的不动元素,轨道数(记作$|X/G|$)由如下公式给出:
			$|X/G| = \frac{1}{|G|}\sum_{g \in G}|X^g|.\,$
	\subsection{五边形数定理}
		设$p(n)$是$n$的拆分数,有$p(n) = \sum_{k \in \mathbb{Z} \setminus \{0\}} (-1)^{k - 1} p\left(n - \frac{k(3k - 1)}{2}\right)$
	\subsection{树的计数}
		\begin{enumerate}\setlength{\itemsep}{-\itemsep}
			\item 有根树计数:$n+1$个结点的有根树的个数为
				$a_{n+1} = \frac{\sum_{j=1}^{n}{j \cdot a_j \cdot{S_{n, j}}}}{n}$
			其中,
				$S_{n, j} = \sum_{i=1}^{n/j}{a_{n+1-ij}} = S_{n-j, j} + a_{n+1-j}$
			\item 无根树计数:当$n$为奇数时,$n$个结点的无根树的个数为
				$a_n-\sum_{i=1}^{n/2}{a_ia_{n-i}}$
			当$n$为偶数时,$n$个结点的无根树的个数为
				$a_n-\sum_{i=1}^{n/2}{a_ia_{n-i}}+\frac{1}{2}a_{\frac{n}{2}}(a_{\frac{n}{2}}+1)$
			\item $n$个结点的完全图的生成树个数为
				$n^{n-2}$
			\item 矩阵-树定理:
			图$G$由$n$个结点构成,设$\bm{A}[G]$为图$G$的邻接矩阵、$\bm{D}[G]$为图$G$的度数矩阵,
			则图$G$的不同生成树的个数为$\bm{C}[G] = \bm{D}[G] - \bm{A}[G]$的任意一个$n-1$阶主子式的行列式值。
		\end{enumerate}
	\subsection{欧拉公式}
		平面图的顶点个数、边数和面的个数有如下关系:
			$V - E + F = C+ 1$
		\indent 其中,$V$是顶点的数目,$E$是边的数目,$F$是面的数目,$C$是组成图形的连通部分的数目。当图是单连通图的时候,公式简化为:
			$V - E + F = 2$
	\subsection{皮克定理}
		给定顶点坐标均是整点(或正方形格点)的简单多边形,其面积$A$和内部格点数目$i$、边上格点数目$b$的关系:
			$A = i + \frac{b}{2} - 1$
	\subsection{牛顿恒等式}
		设$\prod_{i = 1}^n{(x - x_i)} = a_n + a_{n - 1} x + \dots + a_1 x^{n - 1} + a_0 x^n$
		$p_k = \sum_{i = 1}^n{x_i^k}$
		则$a_0 p_k + a_1 p_{k - 1} + \cdots + a_{k - 1} p_1 + k a_k = 0$\par
		特别地,对于$|\bm{A} - \lambda \bm{E}| = (-1)^n(a_n + a_{n - 1} \lambda + \cdots + a_1 \lambda^{n - 1} + a_0 \lambda^n)$
		有$p_k = Tr(\bm{A}^k)$
	%\section{数论公式}
\section{平面几何公式}
	\subsection{三角形}
		\begin{enumerate}\setlength{\itemsep}{-\itemsep}
			\item 半周长
				$p=\frac{a+b+c}{2}$
			\item 面积
				$S=\frac{a \cdot H_a}{2}=\frac{ab \cdot sinC}{2}=\sqrt{p(p-a)(p-b)(p-c)}$
			\item 中线
				$M_a=\frac{\sqrt{2(b^2+c^2)-a^2}}{2}=\frac{\sqrt{b^2+c^2+2bc \cdot cosA}}{2}$
			\item 角平分线 
				$T_a=\frac{\sqrt{bc \cdot [(b+c)^2-a^2]}}{b+c}=\frac{2bc}{b+c}cos\frac{A}{2}$
			\item 高线
				$H_a=bsinC=csinB=\sqrt{b^2-(\frac{a^2+b^2-c^2}{2a})^2}$
			\item 内切圆半径
				\begin{align*}
					r&=\frac{S}{p}=\frac{arcsin\frac{B}{2} \cdot sin\frac{C}{2}}{sin\frac{B+C}{2}}=4R \cdot sin\frac{A}{2}sin\frac{B}{2}sin\frac{C}{2}\\
					&=\sqrt{\frac{(p-a)(p-b)(p-c)}{p}}=p \cdot tan\frac{A}{2}tan\frac{B}{2}tan\frac{C}{2}
				\end{align*}
			\item 外接圆半径
				$R=\frac{abc}{4S}=\frac{a}{2sinA}=\frac{b}{2sinB}=\frac{c}{2sinC}$
		\end{enumerate}
	\subsection{四边形}
		$D_1, D_2$为对角线,$M$对角线中点连线,$A$为对角线夹角,$p$为半周长
		\begin{enumerate}\setlength{\itemsep}{-\itemsep}
			\item $a^2+b^2+c^2+d^2=D_1^2+D_2^2+4M^2$
			\item $S=\frac{1}{2}D_1D_2sinA$
			\item 对于圆内接四边形
				$ac+bd=D_1D_2$
			\item 对于圆内接四边形
				$S=\sqrt{(p-a)(p-b)(p-c)(p-d)}$
		\end{enumerate}
	\subsection{正$n$边形}
		$R$为外接圆半径,$r$为内切圆半径
		\begin{enumerate}\setlength{\itemsep}{-\itemsep}
			\item 中心角
				$A=\frac{2\pi}{n}$
			\item 内角
				$C=\frac{n-2}{n}\pi$
			\item 边长
				$a=2\sqrt{R^2-r^2}=2R \cdot sin\frac{A}{2}=2r \cdot tan\frac{A}{2}$
			\item 面积
				$S=\frac{nar}{2}=nr^2 \cdot tan\frac{A}{2}=\frac{nR^2}{2} \cdot sinA=\frac{na^2}{4 \cdot tan\frac{A}{2}}$
		\end{enumerate}
	\subsection{圆}
		\begin{enumerate}\setlength{\itemsep}{-\itemsep}
			\item 弧长
				$l=rA$
			\item 弦长
				$a=2\sqrt{2hr-h^2}=2r\cdot sin\frac{A}{2}$
			\item 弓形高
				$h=r-\sqrt{r^2-\frac{a^2}{4}}=r(1-cos\frac{A}{2})=\frac{1}{2} \cdot arctan\frac{A}{4}$
			\item 扇形面积
				$S_1=\frac{rl}{2}=\frac{r^2A}{2}$
			\item 弓形面积
				$S_2=\frac{rl-a(r-h)}{2}=\frac{r^2}{2}(A-sinA)$
		\end{enumerate}
	\subsection{棱柱}
		\begin{enumerate}\setlength{\itemsep}{-\itemsep}
			\item 体积
				$V=Ah$
				$A$为底面积,$h$为高
			\item 侧面积
				$S=lp$
				$l$为棱长,$p$为直截面周长
			\item 全面积
				$T=S+2A$
		\end{enumerate}
	\subsection{棱锥}
		\begin{enumerate}\setlength{\itemsep}{-\itemsep}
			\item 体积
				$V=Ah$
				$A$为底面积,$h$为高
			\item 正棱锥侧面积
				$S=lp$
				$l$为棱长,$p$为直截面周长
			\item 正棱锥全面积
				$T=S+2A$
		\end{enumerate}
	\subsection{棱台}
		\begin{enumerate}\setlength{\itemsep}{-\itemsep}
			\item 体积
				$V=(A_1+A_2+\sqrt{A_1A_2}) \cdot \frac{h}{3}$
				$A_1,A_2$为上下底面积,$h$为高
			\item 正棱台侧面积
				$S=\frac{p_1+p_2}{2}l$
				$p_1,p_2$为上下底面周长,$l$为斜高
			\item 正棱台全面积
				$T=S+A_1+A_2$
		\end{enumerate}
	\subsection{圆柱}
		\begin{enumerate}\setlength{\itemsep}{-\itemsep}
			\item 侧面积
				$S=2\pi rh$
			\item 全面积
				$T=2\pi r(h+r)$
			\item 体积
				$V=\pi r^2h$
		\end{enumerate}
	\subsection{圆锥}
		\begin{enumerate}\setlength{\itemsep}{-\itemsep}
			\item 母线
				$l=\sqrt{h^2+r^2}$
			\item 侧面积
				$S=\pi rl$
			\item 全面积
				$T=\pi r(l+r)$
			\item 体积
				$V=\frac{\pi}{3} r^2h$
		\end{enumerate}
	\subsection{圆台}
		\begin{enumerate}\setlength{\itemsep}{-\itemsep}
			\item 母线
				$l=\sqrt{h^2+(r_1-r_2)^2}$
			\item 侧面积
				$S=\pi(r_1+r_2)l$
			\item 全面积
				$T=\pi r_1(l+r_1)+\pi r_2(l+r_2)$
			\item 体积
				$V=\frac{\pi}{3}(r_1^2+r_2^2+r_1r_2)h$
		\end{enumerate}
	\subsection{球}
		\begin{enumerate}\setlength{\itemsep}{-\itemsep}
			\item 全面积
				$T=4\pi r^2$
			\item 体积
				$V=\frac{4}{3}\pi r^3$
		\end{enumerate}
	\subsection{球台}
		\begin{enumerate}\setlength{\itemsep}{-\itemsep}
			\item 侧面积
				$S=2\pi rh$
			\item 全面积
				$T=\pi(2rh+r_1^2+r_2^2)$
			\item 体积
				$V=\frac{\pi h[3(r_1^2+r_2^2)+h^2]}{6}$
		\end{enumerate}
	\subsection{球扇形}
		\begin{enumerate}\setlength{\itemsep}{-\itemsep}
			\item 全面积
				$T=\pi r(2h+r_0)$
				$h$为球冠高,$r_0$为球冠底面半径
			\item 体积
				$V=\frac{2}{3}\pi r^2h$
		\end{enumerate}
\section{立体几何公式}
	\subsection{球面三角公式}
		设$a, b, c$是边长,$A, B, C$是所对的二面角,
		有余弦定理$cos a = cos b \cdot cos c + sin b \cdot sin c \cdot cos A$
		正弦定理$\frac{sin A}{sin a} = \frac{sin B}{sin b} = \frac{sin C}{sin c}$
		三角形面积是$A + B + C - \pi$
	\subsection{四面体体积公式}
		$U, V, W, u, v, w$是四面体的$6$条棱,$U, V, W$构成三角形,$(U, u), (V, v), (W, w)$互为对棱,
		则$V = \frac{\sqrt{(s - 2a)(s - 2b)(s - 2c)(s - 2d)}}{192 uvw}$
		其中$\left\{\begin{array}{lll}
				a & = & \sqrt{xYZ}, \\
				b & = & \sqrt{yZX}, \\
				c & = & \sqrt{zXY}, \\
				d & = & \sqrt{xyz}, \\
				s & = & a + b + c + d, \\ 
				X & = & (w - U + v)(U + v + w), \\
				x & = & (U - v + w)(v - w + U), \\
				Y & = & (u - V + w)(V + w + u), \\
				y & = & (V - w + u)(w - u + V), \\
				Z & = & (v - W + u)(W + u + v), \\
				z & = & (W - u + v)(u - v + W)
			\end{array}\right.$
\section{积分表}
\newcommand{\ud}{\mathrm{d}}
$\arcsin x \to \frac{1}{\sqrt{1-x^2}}				   $\par
$\arccos x \to -\frac{1}{\sqrt{1-x^2}}				  $\par
$\arctan x \to \frac{1}{1+x^2}						  $\par
$a^x \to \frac{a^x}{\ln a}							  $\par
$\sin x \to -\cos x									 $\par
$\cos x \to \sin x									  $\par
$\tan x \to -\ln\cos x								  $\par
$\sec x \to \ln\tan(\frac{x}{2}+\frac{\pi}{4})		  $\par
$\tan^2 x \to \tan x - x								$\par
$\csc x \to \ln\tan\frac{x}{2}						  $\par
$\sin^2 x \to \frac{x}{2} - \frac{1}{2}\sin x\cos x	 $\par
$\cos^2 x \to \frac{x}{2} + \frac{1}{2}\sin x\cos x	 $\par
$\sec^2 x \to \tan x									$\par
$\frac{1}{\sqrt{a^2-x^2}} \to \arcsin\frac{x}{a}		$\par
$\csc^2 x \to -\cot x								   $\par
$\frac{1}{a^2-x^2}(|x|<|a|) \to \frac{1}{2a}\ln\frac{a+x}{a-x}  $\par
$\frac{1}{x^2-a^2}(|x|>|a|) \to \frac{1}{2a}\ln\frac{x-a}{x+a}  $\par
$\sqrt{a^2-x^2} \to \frac{x}{2}\sqrt{a^2-x^2}+\frac{a^2}{2}\arcsin\frac{x}{a}   $\par
$\frac{1}{\sqrt{x^2+a^2}} \to \ln(x+\sqrt{a^2+x^2}) $\par
$\sqrt{a^2+x^2} \to \frac{x}{2}\sqrt{a^2+x^2}+\frac{a^2}{2}\ln(x+\sqrt{a^2+x^2})$\par
$\frac{1}{\sqrt{x^2-a^2}} \to \ln(x+\sqrt{x^2-a^2})$\par
$\sqrt{x^2-a^2} \to \frac{x}{2}\sqrt{x^2-a^2}-\frac{a^2}{2}\ln(x+\sqrt{x^2-a^2})$\par
$\frac{1}{x\sqrt{a^2-x^2}} \to -\frac{1}{a}\ln\frac{a+\sqrt{a^2-x^2}}{x}$\par
$\frac{1}{x\sqrt{x^2-a^2}} \to \frac{1}{a}\arccos\frac{a}{x}$\par
$\frac{1}{x\sqrt{a^2+x^2}} \to -\frac{1}{a}\ln\frac{a+\sqrt{a^2+x^2}}{x}$\par
$\frac{1}{\sqrt{2ax-x^2}} \to \arccos(1-\frac{x}{a})$\par
$\frac{x}{ax+b} \to \frac{x}{a}-\frac{b}{a^2}\ln(ax+b)$\par
$\sqrt{2ax-x^2} \to \frac{x-a}{2}\sqrt{2ax-x^2}+\frac{a^2}{2}\arcsin(\frac{x}{a}-1)$\par
$\frac{1}{x\sqrt{ax+b}}(b<0) \to \frac{2}{\sqrt{-b}}\arctan\sqrt{\frac{ax+b}{-b}}$\par
$x\sqrt{ax+b} \to \frac{2(3ax-2b)}{15a^2}(ax+b)^{\frac{3}{2}}$\par
$\frac{1}{x\sqrt{ax+b}}(b>0) \to \frac{1}{\sqrt{b}}\ln\frac{\sqrt{ax+b}-\sqrt{b}}{\sqrt{ax+b}+\sqrt{b}}$\par
$\frac{x}{\sqrt{ax+b}} \to \frac{2(ax-2b)}{3a^2}\sqrt{ax+b}$\par
$\frac{1}{x^2 \sqrt{ax+b}} \to -\frac{\sqrt{ax+b}}{bx}-\frac{a}{2b}\int\frac{\ud x}{x\sqrt{ax+b}}$\par
$\frac{\sqrt{ax+b}}{x} \to 2\sqrt{ax+b}+b\int\frac{\ud x}{x\sqrt{ax+b}}$\par
$\frac{1}{\sqrt{(ax+b)^n}}(n>2) \to \frac{-2}{a(n-2)}\cdot\frac{1}{\sqrt{(ax+b)^{n-2} }}$\par
$\frac{1}{ax^2+c}(a>0,c>0) \to \frac{1}{\sqrt{ac}}\arctan{(x\sqrt{\frac{a}{c}})}$\par
$\frac{x}{ax^2+c} \to \frac{1}{2a}\ln(ax^2+c)$\par
$\frac{1}{ax^2+c}(a+,c-) \to \frac{1}{2\sqrt{-ac}}\ln\frac{x\sqrt{a}-\sqrt{-c}}{x\sqrt{a}+\sqrt{-c}}$\par
$\frac{1}{x(ax^2+c)} \to \frac{1}{2c}\ln\frac{x^2}{ax^2+c}$\par
$\frac{1}{ax^2+c}(a-,c+) \to \frac{1}{2\sqrt{-ac}}\ln\frac{\sqrt{c}+x\sqrt{-a}}{\sqrt{c}-x\sqrt{-a}}$\par
$x{\sqrt{ax^2+c}} \to \frac{1}{3a}\sqrt{(ax^2+c)^3}$\par
$\frac{1}{(ax^2+c)^n}(n>1) \to \frac{x}{2c(n-1)(ax^2+c)^{n-1}}+\frac{2n-3}{2c(n-1)}\int\frac{\ud x}{(ax^2+c)^{n-1}}$\par
$\frac{x^n}{ax^2+c}(n\ne 1)\to \frac{x^{n-1}}{a(n-1)}-\frac{c}{a}\int\frac{x^{n-2}}{ax^2+c}\ud x$\par
$\frac{1}{x^2(ax^2+c)} \to \frac{-1}{cx}-\frac{a}{c}\int\frac{\ud x}{ax^2+c}$\par
$\frac{1}{x^2(ax^2+c)^n}(n\ge 2) \to \frac{1}{c}\int\frac{\ud x}{x^2(ax^2+c)^{n-1}}-\frac{a}{c}\int\frac{\ud x}{(ax^2+c)^n}$\par
$\sqrt{ax^2+c}(a>0) \to \frac{x}{2}\sqrt{ax^2+c}+\frac{c}{2\sqrt{a}}\ln(x\sqrt{a}+\sqrt{ax^2+c})$\par
$\sqrt{ax^2+c}(a<0) \to \frac{x}{2}\sqrt{ax^2+c}+\frac{c}{2\sqrt{-a}}\arcsin(x\sqrt{\frac{-a}{c}})$\par
$\frac{1}{\sqrt{ax^2+c}}(a>0) \to \frac{1}{\sqrt{a}}\ln(x\sqrt{a}+\sqrt{ax^2+c})$\par
$\frac{1}{\sqrt{ax^2+c}}(a<0) \to \frac{1}{\sqrt{-a}}\arcsin(x\sqrt{-\frac{a}{c}})$\par
$\sin^2 ax \to \frac{x}{2}-\frac{1}{4a}\sin 2ax$\par
$\cos^2 ax \to \frac{x}{2}+\frac{1}{4a}\sin 2ax$\par
$\frac{1}{\sin ax} \to \frac{1}{a}\ln\tan\frac{ax}{2}$\par
$\frac{1}{\cos^2 ax} \to \frac{1}{a}\tan ax$\par
$\frac{1}{\cos ax} \to \frac{1}{a}\ln \tan(\frac{\pi}{4}+\frac{ax}{2})$\par
$\ln(ax)\to x\ln(ax)-x$\par
$\sin^3 ax \to \frac{-1}{a}\cos ax+\frac{1}{3a}\cos^3 ax$\par
$\cos^3 ax \to \frac{1}{a}\sin ax - \frac{1}{3a}\sin^3 ax$\par
$\frac{1}{\sin^2 ax}\to -\frac{1}{a}\cot ax$\par
$x\ln(ax)\to \frac{x^2}{2}\ln(ax)-\frac{x^2}{4}$\par
$\cos ax\to \frac{1}{a}\sin ax$\par
$x^2 e^{ax} \to \frac{e^{ax}}{a^3}(a^2x^2-2ax+2)$\par
$(\ln(ax))^2 \to x(\ln(ax))^2-2x\ln(ax)+2x$\par
$x^2\ln(ax) \to \frac{x^3}{3}\ln(ax)-\frac{x^3}{9}$\par
$x^n\ln(ax) \to \frac{x^{n+1}}{n+1}\ln(ax)-\frac{x^{n+1}}{(n+1)^2}$\par
$\sin(\ln ax) \to \frac{x}{2}[\sin(\ln ax) - \cos(\ln ax)]$\par
$\cos(\ln ax) \to \frac{x}{2}[\sin(\ln ax) + \cos(\ln ax)]$\par
