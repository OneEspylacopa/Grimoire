\begin{lstlisting}
#include <cstdio>
#include <cstdlib>
#include <cstring>
#include <iostream>
#include <algorithm>
using namespace std;
const int N=250;
int n;//点数(1->n)
int head;
int tail;
int Start;
int Finish;
int match[N];//表示哪个点匹配了哪个点
int Father[N];//增广路的Father
int Base[N];//该点属于哪朵花
int Q[N];
bool mark[N];
bool mat[N][N];//邻接矩阵
bool InBlossom[N];
bool in_Queue[N];

void BlossomContract(int x,int y){
	memset(mark,0,sizeof(mark));
	memset(InBlossom,0,sizeof(InBlossom));
#define pre Father[match[i]]
	int lca,i;
	for (i=x;i;i=pre) {i=Base[i]; mark[i]=true; }
	for (i=y;i;i=pre) {i=Base[i]; if (mark[i]) {lca=i; break;} }//寻找lca,一定要注意i=Base[i]
	for (i=x;Base[i]!=lca;i=pre){
		if (Base[pre]!=lca) Father[pre]=match[i];//对于BFS树中的父边是匹配边的点,Father向后跳
		InBlossom[Base[i]]=true;
		InBlossom[Base[match[i]]]=true;
	}
	for (i=y;Base[i]!=lca;i=pre){
		if (Base[pre]!=lca) Father[pre]=match[i];//同理
		InBlossom[Base[i]]=true;
		InBlossom[Base[match[i]]]=true;
	}
#undef pre
	if (Base[x]!=lca) Father[x]=y;//注意不能从lca这个奇环的关键点跳回来
	if (Base[y]!=lca) Father[y]=x;
	for (i=1;i<=n;i++)
		if (InBlossom[Base[i]]){
			Base[i]=lca;
			if (!in_Queue[i]){
				Q[++tail]=i;
				in_Queue[i]=true;//要注意如果本来连向BFS树中父结点的边是非匹配边的点,可能是没有入队的
			}
		}
}

void Change(){
	int x,y,z;
	z=Finish;
	while (z){
		y=Father[z];
		x=match[y];
		match[y]=z;
		match[z]=y;
		z=x;
	}
}

void FindAugmentPath(){
	memset(Father,0,sizeof(Father));
	memset(in_Queue,0,sizeof(in_Queue));
	for (int i=1;i<=n;i++) Base[i]=i;
	head=0; tail=1;
	Q[1]=Start;
	in_Queue[Start]=1;
	while (head!=tail){
		int x=Q[++head];
		for (int y=1;y<=n;y++)
			if (mat[x][y] && Base[x]!=Base[y] && match[x]!=y) {//无意义的边
				if ( Start==y || (match[y] && Father[match[y]]))//精髓地用Father表示该点是否
					BlossomContract(x,y);
				else if (!Father[y]){
					Father[y]=x;
					if (match[y]){
						Q[++tail]=match[y];
						in_Queue[match[y]]=true;
					}
					else{
						Finish=y;
						Change();
						return;
					}
				}
			}
	}
}

void Edmonds(){
	memset(match,0,sizeof(match));
	for (Start=1;Start<=n;Start++)
		if (match[Start]==0)
			FindAugmentPath();
}

void output(){
	memset(mark,0,sizeof(mark));
	int cnt=0;//一般图最大匹配  最大点数
	for (int i=1;i<=n;i++)
		if (match[i]) cnt++;
	printf("%d\n",cnt);
	for (int i=1;i<=n;i++)
		if (!mark[i] && match[i]){
			mark[i]=true;//i和match[i]匹配
			mark[match[i]]=true;
			printf("%d %d\n",i,match[i]);
		}
}

int main(){
	int x,y;
	scanf("%d",&n);
	memset(mat,0,sizeof(mat));
	while (scanf("%d%d",&x,&y)!=EOF)
		mat[x][y]=mat[y][x]=1;
	Edmonds();
	output();
	return 0;
}
\end{lstlisting}
