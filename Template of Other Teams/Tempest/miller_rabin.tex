miller\_rabin\_32是针对32位以下整数的; miller\_rabin\_64是针对64位以下整数的.
直接调用prime()函数, 当返回值是true时表示是素数, 否则不是质数.
\begin{lstlisting}
namespace miller_rabin_32 {
	int const n = 3;
	int const base[] = {2, 7, 61};
 
	inline long long power(int x, int k, int p) {
		long long ans = 1, num = x % p;
		for (int i = k; i > 0; i >>= 1) {
			if (i & 1) {
				(ans *= num) %= p;
			}
			(num *= num) %= p;
		}
		return ans;
	}
 
	inline bool check(int p, int base) {
		int n = p - 1;
		while (!(n & 1)) {
			n >>= 1;
		}
		long long m = power(base, n, p);
		while (n != p - 1 && m != 1 && m != p - 1) {
			(m *= m) %= p;
			n <<= 1;
		}
		return m == p - 1 || (n & 1) == 1);
	}
 
	inline bool prime(int p) {
		for (int i = 0; i < n; ++i) {
			if (base[i] == p) {
				return true;
			}
		}
		if (p == 1 || !(p & 1)) {
			return false;
		}
		for (int i = 0; i < n; ++i) {
			if (!check(p, base[i])) {
				return false;
			}
		}
		return true;
	}
}
 
namespace miller_rabin_64 {
	int const n = 9;
	int const base[] = {2, 3, 5, 7, 11, 13, 17, 19, 23};
 
	inline long long multiply(const long long &x, const long long &y, const long long &p) {
		long long ans = 0, num = x % p;
		for (long long i = y; i > 0; i >>= 1) {
			if (i & 1) {
				(ans += num) %= p;
			}
			(num += num) %= p;
		}
		return ans;
	}
 
	inline long long power(const long long &x, const long long &k, const long long &p) {
		long long ans = 1, num = x % p;
		for (long long i = k; i > 0; i >>= 1) {
			if (i & 1) {
				ans = multiply(ans, num, p);
			}
			num = multiply(num, num, p);
		}
		return ans;
	}
 
	inline bool check(const long long &p, const long long &base) {
		long long n = p - 1;
		while (!(n & 1)) {
			n >>= 1;
		}
		long long m = power(base, n, p);
		while (n != p - 1 && m != 1 && m != p - 1) {
			m = multiply(m, m, p);
			n <<= 1;
		}
		return m == p - 1 || (n & 1) == 1;
	}
 
	inline bool prime(const long long &p) {
		for (int i = 0; i < n; ++i) {
			if (base[i] == p) {
				return true;
			}
		}
		if (p == 1 || !(p & 1)) {
			return false;
		}
		for (int i = 0; i < n; ++i) {
			if (!check(p, base[i])) {
				return false;
			}
		}
		return true;
	}
}
\end{lstlisting}
