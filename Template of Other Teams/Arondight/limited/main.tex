%!TEX program = xelatex
\documentclass[landscape, oneside, a4paper, cs4size]{book}

\def\marginset#1#2{                      % 页边设置 \marginset{left}{top}
\setlength{\oddsidemargin}{#1}         % 左边(书内侧)装订预留空白距离
\iffalse                   % 如果考虑左侧(书内侧)的边注区则改为\iftrue
\reversemarginpar
\addtolength{\oddsidemargin}{\marginparsep}
\addtolength{\oddsidemargin}{\marginparwidth}
\fi
\setlength{\evensidemargin}{0mm}       % 置0
\iffalse                   % 如果考虑右侧(书外侧)的边注区则改为\iftrue
\addtolength{\evensidemargin}{\marginparsep}
\addtolength{\evensidemargin}{\marginparwidth}
\fi
% \paperwidth = h + \oddsidemargin+\textwidth+\evensidemargin + h
\setlength{\hoffset}{\paperwidth}
\addtolength{\hoffset}{-\oddsidemargin}
\addtolength{\hoffset}{-\textwidth}
\addtolength{\hoffset}{-\evensidemargin}
\setlength{\hoffset}{0.5\hoffset}
\addtolength{\hoffset}{-1in}           % h = \hoffset + 1in
%\setlength{\voffset}{-1in}             % 0 = \voffset + 1in
\setlength{\topmargin}{\paperheight}
\addtolength{\topmargin}{-\headheight}
\addtolength{\topmargin}{-\headsep}
\addtolength{\topmargin}{-\textheight}
\addtolength{\topmargin}{-\footskip}
\addtolength{\topmargin}{#2}           % 上边预留装订空白距离
\setlength{\topmargin}{0.5\topmargin}
}
% 调整页边空白使内容居中,两参数分别为纸的左边和上边预留装订空白距离
\marginset{125mm}{200mm}


%\usepackage{ctex}
\usepackage{bm}
%\usepackage[fleqn]{amsmath}
\usepackage{harpoon}
\usepackage{fontspec}
\usepackage{listings}
\usepackage[left=1cm,right=1cm,top=1.2cm,bottom=1cm,columnsep=1cm,dvipdfm]{geometry}
\usepackage{setspace}
\usepackage{bm}
\usepackage{cmap}
\usepackage{cite}
\usepackage{float}
\usepackage{xeCJK}
\usepackage{amsthm}
\usepackage{amsmath}
\usepackage{amssymb}
\usepackage{multirow}
\usepackage{multicol}
\usepackage{setspace}
\usepackage{enumerate}
\usepackage{indentfirst}
\usepackage{adjmulticol}
\usepackage{titlesec}
\usepackage{color,minted}
\usepackage{xeCJK}
\allowdisplaybreaks
%\setlength{\parindent}{0em}
%\setlength{\mathindent}{0pt}
\lstset{breaklines}
\let\cleardoublepage\relax
\titleformat{\chapter}{\normalfont\normalsize\sffamily}{\thechapter}{10pt}{}
\titleformat{\section}{\normalfont\footnotesize\sffamily}{\thesection}{1em}{}
\titleformat{\subsection}{\normalfont\footnotesize\sffamily}{\thesubsection}{1em}{}
\titleformat{\subsubsection}{\normalfont\footnotesize\sffamily}{\thesubsubsection}{1em}{}
\titlespacing*{\chapter} {0pt}{5pt}{5pt}
\titlespacing*{\section} {0pt}{0pt}{0pt}
\titlespacing*{\subsection} {0pt}{0pt}{0pt}
\titlespacing*{\subsubsection}{0pt}{0pt}{0pt}
%configure fonts
\setmonofont{LMMono10-Regular}[Scale=0.8]
%\setmonofont{FiraCode-Retina}[Scale=0.8]
\setCJKmainfont{FandolSong-Regular}
\setCJKsansfont{SourceHanSans-Medium}
\setCJKmonofont[Scale=0.8]{STXihei}
\usepackage{yfonts}

\usepackage{fancyhdr}

\renewcommand{\theFancyVerbLine}{\sffamily \textcolor[rgb]{0.5,0.5,0.5}{\scriptsize {\arabic{FancyVerbLine}}}}

\usemintedstyle{tango}

\setminted[cpp]{
	style=xcode,
	mathescape,
	linenos,
	autogobble,
	baselinestretch=0.8,
	tabsize=2,
	fontsize=\normalsize,
	%bgcolor=Gray,
	frame=single,
	framesep=1mm,
	framerule=0.3pt,
	numbersep=1mm,
	breaklines=true,
	breaksymbolsepleft=2pt,
	%breaksymbolleft=\raisebox{0.8ex}{ \small\reflectbox{\carriagereturn}}, %not moe!
	%breaksymbolright=\small\carriagereturn,
	breakbytoken=false,
}
\setminted[java]{
	style=xcode,
	mathescape,
	linenos,
	autogobble,
	baselinestretch=0.8,
	tabsize=2,
	fontsize=\normalsize,
	%bgcolor=Gray,
	frame=single,
	framesep=1mm,
	framerule=0.3pt,
	numbersep=1mm,
	breaklines=true,
	breaksymbolsepleft=2pt,
	%breaksymbolleft=\raisebox{0.8ex}{ \small\reflectbox{\carriagereturn}}, %not moe!
	%breaksymbolright=\small\carriagereturn,
	breakbytoken=false,
}
\setminted[text]{
	style=xcode,
	mathescape,
	linenos,
	autogobble,
	baselinestretch=0.8,
	tabsize=4,
	fontsize=\normalsize,
	%bgcolor=Gray,
	frame=single,
	framesep=1mm,
	framerule=0.3pt,
	numbersep=1mm,
	breaklines=true,
	breaksymbolsepleft=2pt,
	%breaksymbolleft=\raisebox{0.8ex}{ \small\reflectbox{\carriagereturn}}, %not moe!
	%breaksymbolright=\small\carriagereturn,
	breakbytoken=false,
}

\usepackage{lastpage}
\pagestyle{fancy}
\fancypagestyle{plain}{}
\fancyhf{}
\lhead{Shanghai Jiao Tong University × Arondight}
\chead{\leftmark}
\rhead{\thepage/\pageref{LastPage}}
\setlength{\headsep}{1pt}

\usepackage{tocloft}
\makeatletter
\renewcommand{\@cftmaketoctitle}{}
\makeatother

\usepackage{punk}

\begin{document}\scriptsize
	\renewcommand{\thefootnote}{\fnsymbol{footnote}}
	\title{\Huge{{\punkfamily Arondight's Standard Code Library}}\thanks{https://www.github.com/footoredo/Arondight}}
	\author{\emph{Shanghai Jiao Tong University}}
	\date{Dated: \today}
	\maketitle
	\clearpage
	\begin{multicols}{2}
		\tableofcontents
		\clearpage
		\begin{spacing}{0.8}
			\def \source {../source}
\chapter{代数}
\section{$O(n^2\log n)$求线性递推数列第n项}
Given $a_0, a_1, \cdots , a_{m-1} \\
\indent a_n = c_0 * a_{n-m} + \cdots + c_{m-1} * a_0 \\
\indent a_0 \ is \ the \ nth \ element, \cdots, a_{m-1} \ is \ the \ n+m-1th \ element
$
\inputminted{cpp}{\source/algebra/linear-recursion.cpp}
\section{闪电数论变换与魔力CRT}
\inputminted{cpp}{\source/algebra/NTT+CRT.cpp}
\section{多项式求逆}
Given polynomial a and n, b is the polynomial such that $a * b \equiv 1 (\mod x^n) $
\inputminted{cpp}{\source/algebra/polynomial-inverse.cpp}
\section{多项式除法}
d is quotient and r is remainder
\inputminted{cpp}{\source/algebra/polynomial-divide.cpp}
\section{多项式取指数取对数}
Given polynomial a and n, b is the polynomial such that $b \equiv e^a (\mod x^n)$ or $b \equiv \ln a (\mod x^n)$
\inputminted{cpp}{\source/algebra/polynomial-expandln.cpp}
\section{快速沃尔什变换}
\inputminted{cpp}{\source/algebra/FWT.cpp}
\chapter{数论}
\section{大整数相乘取模}
\inputminted{cpp}{\source/number-theory/biginteger-multiply.cpp}
\section{线段下整点}
solve for $\sum_{i=0}^{n-1} \lfloor \frac{a+bi}{m}\rfloor$, $n,m,a,b>0$
\inputminted{cpp}{\source/number-theory/integer-lattice-under-segment.cpp}
\section{中国剩余定理}
first is remainder, second is module
\inputminted{cpp}{\source/number-theory/CRT.cpp}
\chapter{图论}
\section{一般图匹配}
\inputminted{cpp}{\source/graph-theory/general-matching.cpp}
\section{一般最大权匹配}
\inputminted{cpp}{\source/graph-theory/weighted_blossom.cpp}
\section{无向图最小割}
\inputminted{cpp}{\source/graph-theory/StoerWagner_O(V^3).cpp}
\chapter{字符串}
\section{Manacher}
\inputminted{cpp}{\source/string/manacher.cpp}
\section{指针版回文自动机}
\inputminted{cpp}{\source/string/PalindromeAutomaton_pointer.cpp}
\section{数组版后缀自动机}
\inputminted{cpp}{\source/string/SuffixAutomaton_array.cpp}
\section{指针版后缀自动机}
\inputminted{cpp}{\source/string/SuffixAutomaton_pointer.cpp}
\section{广义后缀自动机}
\inputminted{cpp}{\source/string/EX_SuffixAutomaton_pointer.cpp}
\section{后缀数组}
\inputminted{cpp}{\source/string/SA.cpp}
\section{最小表示法}
\inputminted{cpp}{\source/string/min_express.cpp}
\chapter{技巧}
\section{无敌的读入优化}
\inputminted{cpp}{\source/hints/input-acceleration.cpp}
\section{真正释放STL内存}
\inputminted{cpp}{\source/hints/STL-memory-release.cpp}
		\end{spacing}
	\end{multicols}
\end{document}
